% Chapter 2

\chapter{Corpus and patterns} % Chapter title

\label{ch:patterns} % For referencing the chapter elsewhere, use \autoref{ch:examples} 


\section{Sources for emotion-triggering patterns}

Don't use ambiguous ones (e.g. anxious can mean both eager and worried)
Intuitively, we would want to employ mainly transitive constructions in which the subject denotes the experiencer and the object refers to the cause, also constructions in which subject is experiencer and clause denotes object


\subsection{Introspection}

We derive initial patterns from introspection and making use of online dictionaries and thesauri.

\subsection{Emotions in VerbNet, FrameNet}
VerbNet is not really useful because we don't care about particular verbs, but emotions and propositions. VerbNet doesn't differentiate between different emotions only between different alternations, e.g. "fear" is a member of the "admire" class.

--> marvel at?

FrameNet has frames for
fear
lexical units that create this frame
afraid.a, apprehension.n, dread.n, fear.n, freaked.a, frightened.a, $live_in_fear$.v, nervous.a, scared.a, terrified.a, terror.n

trust
lexical units: believe.v, credence.n, credulous.a, faith.n, gullible.a, reliability.n, reliable.a, trust.n, trust.v, trustworthy.a

$cause_emotion$
lexical units: affront.n, affront.v, $call_names$.v, concern.v, insult.n, insult.v, offend.v, offense.n, offensive.a



\subsection{Emotion verb classes from Mathieu and Fellbaum}
A corpus-based construction of emotion verb classes
reread 1. Introduction and 2. Emotion verbs

surprise: astonish, surprise, amaze, astound, strike, stun, floor, dumbfounded, flabbergasted, stupefy
fear: intimidate, scare, frighten, alarm, terrify



\subsection{Dictionaries and thesauri}

Oxford English Dictionary
Most of the time, first sense was taken
\begin{itemize}
	\item Joy: \textit{A vivid emotion of pleasure arising from a sense of well-being or satisfaction; the feeling or state of being highly pleased or delighted; exultation of spirit; gladness, delight.}
	\item Trust: \textit{Confidence in or reliance on some quality or attribute of a person or thing, or the truth of a statement. Const. in (of, on, upon, to, unto).}
	\item Fear: \textit{The emotion of pain or uneasiness caused by the sense of impending danger, or by the prospect of some possible evil.}
	\item Surprise: \textit{The feeling or mental state, akin to astonishment and wonder, caused by an unexpected occurrence or circumstance.} first sense is act of surprise, military act, etc.
	\item Sadness: \textit{The condition or quality of being sad (Of a person, or his or her feelings, disposition, etc.: feeling sorrow; sorrowful, mournful, heavy-hearted.).} Obsolete senses precede it.
	\item Disgust: \textit{Strong repugnance, aversion, or repulsion excited by that which is loathsome or offensive, as a foul smell, disagreeable person or action, disappointed ambition, etc.; profound instinctive dislike or dissatisfaction.}
	\item Anger: \textit{The active feeling provoked against the agent; passion, rage; wrath, ire, hot displeasure.}
	\item Anticipation: \textit{The action of looking forward to, expectation.}
\end{itemize}
a dozen words

The Free Dictionary

a dozen words

Merriam-Webster
synonyms for verbs 
anger: enrage, incense, inflame (also enflame), infuriate, ire, madden, outrage, rankle, rile, roil, steam up, tick off
fear: bother, fear, fret, fuss, stew, stress, sweat, trouble
joy (rejoice): crow, delight, exuberate, glory, jubilate, joy, kvell, rejoice, triumph
sadness (sadden): bum (out), burden, dash, deject, get down, oppress, sadden, weigh down
anticipation (anticipate): anticipate, await, hope (for), watch (for)
disgust: gross out, nauseate, put off, repel, repulse, revolt, sicken, turn off



Roget's Thesaurus (thesaurus.com)
as used for target terms in Crowdsourcing a Word-Emotion Association Lexicon
chosen verbs with highest relevance
anticipate: expect, predict, assume, await, count on, forecast, foresee, prepare for, see
anger: aggravate, annoy, antagonize, arouse, displease, embitter, enrage, exacerbate, exasperate, excite, incense, inflame, infuriate, irritate, offend, outrage, provoke, rankle, rile
fear: feel alarmed, be scared off, anticipate, avoid, dread, expect, foresee, shun, suspect, worry
joy: exult, revel, make happy, delight, amuse, attract, charm, cheer, enchant, enrapture, entertain, fascinate, gratify, please, rejoice, satisfy, thrill, wow
sadness (sadden): discourage, dishearten, dispirit, grieve
disgust: bother, disenchant, displease, disturb, insult, irk, nauseate, offend, outrage, revolt, shock, sicken, turn off, upset
surprise: astonish, amaze, astound, awe, bewilder, confound, confuse, dazzle, disconcert, dismay, dumbfound, flabbergast, overwhelm, perplex, rattle, shock, startle, stun, unsettle
trust: count on, depend on, look to



\subsection{Sentiment lexica (highly associated verbs)}

Harvard General Inquirer
Pstv 1045 positive words, an earlier version of Positiv.
A subset of 557 words are also tagged Affil for words indicating affiliation or supportiveness.
Ngtv 1160 negative words, an earlier version of Negativ.
A subset of 833 words are also tagged Hostile for words indicating an attitude or concern with hostility or aggressiveness.
Pleasur168 words indicating the enjoyment of a feeling, including words indicating confidence, interest and commitment.
Pain 254 words indicating suffering, lack of confidence, or commitment.
Feel 49 words describing particular feelings, including gratitude, apathy, and optimism, not those of pain or pleasure.
-- not really ordered, useful for our purposes
Arousal 166 words indicating excitation, aside from pleasures or pains, but including arousal of affiliation and hostility.
EMOT 311 words related to emotion that are used as a disambiguation category, but also available for general use.
--> have notes for some adjective: http://www.wjh.harvard.edu/~inquirer/EMOT.html
joy; admire; adore; appreciate; be fond of
trust
fear
surprise
sadness
disgust
anger
anticipation

WordAffectLexicon



EmoLex (crowd-sourced word-emotion association lexicon)


SentiWordNet
Helpful, doesn't yield any additional insights, though


\subsection{FrameNet}

FrameNet has a frame hierarchy and frame-to-frame relations.

In FrameNet emotions are conceptualized in the Emotions frame, which describes an Experiencer in a particular emotional State that was provoked by a Stimulus.

The frame Emotions is used by 10 other frames. Of these, we investigate more closely the Desiring, Emotion\_active, Emotion\_directed, and Experiencer\_obj frames.

In the Desiring frame, the Experiencer desires that an Event occur. We derive 20 verbs from the Desiring frame, all of whom we label with anticipation.

In the Experiencer\_obj frame, some phenomenon (the Stimulus) provokes a particular emotion in the Experiencer. Experiencer and stimulus (cause) are core frame elements, making this frame ideal for our purposes. We derive 132 verbs from this frame.

The Emotion\_active frame is similar to Experiencer\_obj, but in this frame the verbs are more active. We derive 9 verbs from this frame.

Finally, the Emotion\_directed frame focuses on adjectives and nouns describing an Experiencer's emotional response to a Stimulus. We keep a few of these, convert some into verbs, and discard the rest, resulting in 12 verbs and adjectives from this frame.

Furthermore, Emotions is inherited by the Emotions\_by\_stimulus frame that is in turn inherited by the Annoyance, Emotions\_by\_possibility, i.e. Fear, Emotions\_of\_mental\_activity, Emotions\_of\_success\_or\_failure, Just\_found\_out, i.e. surprise, Others\_situation\_as\_stimulus frames.

While these frames all refer to emotions, we have obtained most of their lexical entries already from other sources and thus don't make use of them.


\subsection{WordNet}

We make use of WordNet-Affect \cite{wordnet-affect}, an additional hierarchy of "affective" domain labels, which annotates WordNet synsets representing affective concepts in the semi-automatically augmented WordNet Domains further with affective labels. WordNet-Affect-1.1 only contains synsets that were tagged with the label "emo(tion)" in the previous version and thus should serve our purposes. Given the ID listed in the WordNet-Affect-1.1 synsets, we retrieve the respective WordNet-3.0 synset lemmas using the NLTK toolkit\cite{nltk} and the corresponding category.

Adjectives, verbs, and adverbs are related to nouns. For these, the category of the corresponding noun is retrieved. We end up with 279 distinct categories, which we map to Plutchik's 8 emotions.


\subsection{Adjectives}

- Adjectives are more indicate of emotion than nouns or verbs (see also what Anette said)

- In WordNet, nouns and verbs are clustered in synsets and supersenses. No taxonomic hierarchy for adjectives. \cite{adjective_supersenses} induce supersenses for adjectives taking GermaNet's guidelines\footnote{http://www.sfs.uni-tuebingen.de/lsd/adjectives.shtml} as inspiration.

- They build a weakly supervised classifier that labels adjective types (irrespective of context), which they train on a small set of seed examples, some of them translations of GermaNet. We take the translations that pertain to FEELING and manually label them with Plutchik's 8 emotion classes. Furthermore, they released 7511 WordNet adjectives tagged by their classifier with a an adjective type vector. From these we derive the 920 adjectives that are labeled with the adjective type FEELING; we take the label as indicative of the emotion, as the classifier has a high accuracy (k-4 accuracy of 91\%).

- The 920 adjectives contain instances like unknown, unreal, etc. and thus prove not useful.

- We instead take the 126 adjectives. 


\section{Compilation}

From the previously mentioned sources, we derive now the pattern templates. We list the pattern in lemma form along with its emotion and its Stanford part-of-speech tags. We also list information about the constituent of the cause of emotion, if it is an NP or an S. Finally, we list if a passive form exists that can be used as an additional pattern.

We convert the pattern templates into regular expressions that can be matched against the Gigaword corpus. We allow adjectives to be modified, but exclude negation.

\section{Selection of corpus}

default choice: news corpus of Gigaword
not too rich in reporting emotions other than anger and expectation
genre could clearly play a role.
Both thematically (which is more domain than genre) and stylistically (this is what is really meant by genre)
e.g. if working on novels, we'd certainly get a wider variety of emotional expressions.

GW contains the NYT corpus that is annotated for domains. 
Eva Sourjikova analyzed this and extracted specific domain sub-corpora. 

Test whether applying your extractions on categories like: Opinion/Letters; MentalHealthDisorders; NY City; Child abuse and neglect; freedom and human rights, etc. are more prone for more varied emotions.

!!Pre-select domains that are emotion-prone by checking which of them have a high rate of emotion words from those that you pre-selected.

Outlook: turn to other corpora of different genres, e.g. ukWaC is richer in adjective meaning variety, this we know from Matthias Hartungs work). There is also a parsed version (pukWaC).

\section{General guidelines}

Our goal: Harvesting
it's not possible to take all, particularly complex/complicated constructions, e.g. multiple embeddings, proverbs, idionsyncracies.

take clear cases, easy structures that can be easily disambiguated

define filter for extraction, take only those patterns, whose proposition is a clear-cut form

recognize passive, normalize diathesis --> how?

in case of too few items, corpus can be bigger; 
--> statistics necessary of how big corpus should be

examples indicate that often that often modifiers, pre-modifiers, etc. are highly relevant for the sentiment
--> generally interesting to save all meaningful dependencies
--> only possible if propositions have a reasonable size
--> estimate length of simple embedded sentences, possibly with/without cutting off of adjuncts (not arguments)

lemma for representational level; also index advisable to be able to go back and check if further normalisations or elements have to be taken care of

part-of-speech labeling is important, as e.g. trust or fear can function both as verbs and nouns.


\section{Important observations}

- only had used Stanford present verb tag label before, inclusion of other labels increased recall from 1/400 to 1/100
- many matches involved the same trigger words; not bad by itself. If unambigious and frequent, this is exactly what we want.
- two interesting aspects related to this:
1: find frequent and unambiguous trigger contexts so that from these we can find indications on what are secondary emotion words in their scope (complements)
2: (which we did not choose as primary step to take if I recall correctly) is to acquire a wider variation of emotion indicating words. This we could still do in a second step.

- both corpus as well as indicating expressions matter (see above)

\subsection{Domain}

According to Eva Sourjikova:
Semantic content of a text / Subject
Domain is determined by the semantic subject of the text mesage. It can be classified under one
or more general topic and into a set of specific sub-topics or also a combination of more topics.
Some words (terms) are more likely to appear for a given domain, so called domain terms. As
a conceptual ontology, we can imagine a hierarchy of domain specific texts. Some possible tasks:
text classification, topic detection, controlled vocabulary. terms exhibit a degree of domain independence. Some terms are more likely to appear for a given topics, while other terms are far more
generic and will appear in almost every topic regardless of how similar the topics are

\section{Remaining questions}

Look at indicating expressions themselves.
Many of categories more related to adjectival or predicative expressions and less clearly related to verb constructions
--> useful to include constructions that involve adjectives, e.g. be glad that; be afraid that; ..

!!paper \citep{adjective_supersenses} induce supersenses for adjectives. One class is emotion. These should be very useful trigger words. And it would be interesting if one can classify them into these 8 emotion classes. 



\section{Next steps}

1. apply current method to some varied subgenres/domains (guided by the NTY categories) and plot the effects on the basis of the classified patterns you have at the moment

2. consider including more word categories.
adjectives: in most cases still able to identify bearer of emotion (ask for advise, if necessary)
nouns: nominalized verbs from NomBank; possibly more matches in standard text genres, more difficult to identify the bearer of the emotion; often implicit
adjectives are most promising

Anette:
When I looked at the sentence examples, and also the cut-down predicates of the embedded sentences, I was wondering whether extracting verb argument structures from the acquired propositions is the right way to go. Maybe it will be possible with a better filtering technique (getting more data - and there should be enough - but cleaning it by better filters). But we could, in a first step, also use the bare embedded material and apply topic modeling to them. This would mean we induce topics for embedded contexts that are pre-labelled with the embedding emotion indicating predicates. This way, we should be able to induce topics that reflect these emotions, and more filtering could be applied then, to sharpen these contexts or topics in better ways. This is related to the work I did with Matthias Hartung \cite{hartung2011exploring}. It's a kind of distant supervision for inducing semantically constrained LDA topics.

But also using some association statistics for detecting collocations could be a first step to see what we can get out of these contexts.


- Possibly more interesting to investigate nature/clarity of Plutchik's emotion classes in contrast to a pure pos/neg/neutral classification. 



%----------------------------------------------------------------------------------------

%\lipsum[1]
%
%%----------------------------------------------------------------------------------------
%
%\section{A New Section}
%
%\lipsum[2]
%
%Examples: \textit{Italics}, \spacedallcaps{All Caps}, \textsc{Small Caps}, \spacedlowsmallcaps{Low Small Caps}\footnote{Footnote example.}.
%
%%------------------------------------------------
%
%\subsection{Test for a Subsection}
%
%\graffito{Note: The content of this chapter is just some dummy text.}
%\lipsum[3-5]
%
%%------------------------------------------------
%
%\subsection{Autem Timeam}
%
%\lipsum[6]
%
%%----------------------------------------------------------------------------------------
%
%\section{Another Section in This Chapter}
%
%\lipsum[7]
%
%Sia ma sine svedese americas. Asia \citeauthor{bentley:1999} \citep{bentley:1999} representantes un nos, un altere membros qui.\footnote{De web nostre historia angloromanic.} Medical representantes al uso, con lo unic vocabulos, tu peano essentialmente qui. Lo malo laborava anteriormente uso.
%
%\begin{description}
%\item[Description-Label Test:] \lipsum[8]
%\item[Label Test 2:] \lipsum[9]
%\end{description}
%
%\noindent This statement requires citation \citeauthor{cormen:2001} \citep{cormen:2001}.
%
%%------------------------------------------------
%
%\subsection{Personas Initialmente}
%
%\lipsum[10]
%
%\subsubsection{A Subsubsection}
%\lipsum[11]
%
%\paragraph{A Paragraph Example} \lipsum[12]
%
%\begin{aenumerate}
%\item Enumeration with small caps
%\item Second item
%\end{aenumerate}
%
%\noindent Another statement requiring citation \citeauthor{sommerville:1992} \citep{sommerville:1992} but this time with text after the citation.
%
%\begin{table}
%\myfloatalign
%\begin{tabularx}{\textwidth}{Xll} \toprule
%\tableheadline{labitur bonorum pri no} & \tableheadline{que vista}
%& \tableheadline{human} \\ \midrule
%fastidii ea ius & germano &  demonstratea \\
%suscipit instructior & titulo & personas \\
%\midrule
%quaestio philosophia & facto & demonstrated \citeauthor{knuth:1976} \\
%\bottomrule
%\end{tabularx}
%\caption[Autem timeam deleniti usu id]{Autem timeam deleniti usu id. \citeauthor{knuth:1976}}  
%\label{tab:example}
%\end{table}
%
%\enlargethispage{2cm}
%
%%------------------------------------------------
%
%\subsection{Figure Citations}
%Veni introduction es pro, qui finalmente demonstrate il. E tamben anglese programma uno. Sed le debitas demonstrate. Non russo existe o, facite linguistic registrate se nos. Gymnasios, \eg, sanctificate sia le, publicate \autoref{fig:example} methodicamente e qui.
%
%Lo sed apprende instruite. Que altere responder su, pan ma, \ie, signo studio. \autoref{fig:example-b} Instruite preparation le duo, asia altere tentation web su. Via unic facto rapide de, iste questiones methodicamente o uno, nos al.
%
%\begin{figure}[bth]
%\myfloatalign
%\subfloat[Asia personas duo.]
%{\includegraphics[width=.45\linewidth]{gfx/example_1}} \quad
%\subfloat[Pan ma signo.]
%{\label{fig:example-b}
%\includegraphics[width=.45\linewidth]{gfx/example_2}} \\
%\subfloat[Methodicamente o uno.]
%{\includegraphics[width=.45\linewidth]{gfx/example_3}} \quad
%\subfloat[Titulo debitas.]
%{\includegraphics[width=.45\linewidth]{gfx/example_4}}
%\caption[Tu duo titulo debitas latente]{Tu duo titulo debitas latente.}\label{fig:example}
%\end{figure}